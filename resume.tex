\documentclass{article}
\usepackage{geometry}
\usepackage{enumitem}
\usepackage{parskip}
\usepackage{hyperref}
\usepackage{graphicx}
\usepackage{titlesec}

\geometry{a4paper, margin=1in}
\hypersetup{colorlinks=true, linkcolor=blue, urlcolor=blue}
\titleformat{\section}{\Large\bfseries}{\thesection}{1em}{}

\begin{document}

\begin{center}
    \textbf{\LARGE Wuxie Li} \\
    \vspace{0.5em}
    \textit{Summary}
\end{center}

\noindent Innovative and detail-oriented student with a strong foundation in statistics and artificial intelligence, demonstrated through rigorous academic performance and hands-on project experience. Possess a solid foundation in data analysis, machine learning, and deep learning frameworks such as PyTorch. Deeply passionate about research and mathematics, I am committed to advancing my expertise in mathematical and statistical methodologies, with a focus on their practical modeling applications and implementation within artificial intelligence.

\section{Core Competences}
\begin{itemize}[leftmargin=1.5em]
    \item Data Analysis \& Statistical Modeling
    \item Basic Knowledge of Machine Learning \& Deep Learning
    \item Python \& C++ Programming
    \item Familiar with Generative Models (GAN, VAE, DDPM, DDIM, etc.)
    \item Data Visualization \& Processing
    \item Problem-Solving \& Critical Thinking
    \item Independent Learning
    \item LaTeX \& Markdown Writing
\end{itemize}

\section{Education}
\begin{tabular}{p{3cm}p{11cm}}
    \textbf{Nankai University} & Engineering Experimental Class \\
    & September 2023 -- August 2024 \\
    & Completed a rigorous fundamental curriculum including High Level Language Program Design 2-1(4.0), Linear Algebra(3.7), Higher Mathematics(3.7), Probability and Mathematical Statistics(3.7). \\
    \\
    \textbf{Nankai University} & School of Statistics and Data Science \\
    & (Major Switch) \\
    & August 2024 -- Present \\
    & Developed strong analytical and computational skills with courses such as Mathematical Analysis I(4.0), Advanced Algebra and Analytic Geometry I(4.0), Theory of Probability(3.3) and Database System(3.3). \\
    & Achieved an overall GPA of 3.55/4.0, ranking in the top 25\% of the class. \\
    & Demonstrated academic excellence in major-specific courses, maintaining a GPA of 3.323/4.0 and ranking 27.6\%.
\end{tabular}

\section{Research Interests}
\begin{itemize}[leftmargin=1.5em]
    \item Deep Learning \& Diffusion Models: Passionate about advancing the field of deep learning with a particular focus on probabilistic modeling.
    \item Technical Expertise: Self-taught in the PyTorch framework with hands-on experience in generative models including GANs, VAEs, and diffusion processes (DDPM, DDIM).
    \item Ongoing Learning: Currently expanding expertise in score-based generative modeling to explore innovative solutions in data-driven problem solving.
\end{itemize}

\section{Project Experience}
\textbf{Interdisciplinary Contest in Modeling (ICM) 2025} \\
\textit{Data Science \& Cybersecurity Analysis} \\
January 23 -- January 27, 2025 \\
\textbf{Project Overview}: Led a data-driven analysis of cybersecurity crime trends by integrating machine learning techniques with policy review, focusing on crime distribution over time and across regions. Proposed a policy paradigm for cybersecurity governance based on data-driven insights.

\begin{itemize}[leftmargin=1.5em]
    \item Conducted extensive data mining from cybersecurity crime databases and national policy documents.
    \item Applied statistical methods such as cluster analysis, principal component analysis, and difference-in-differences (DID) models to identify patterns and trends.
    \item Designed and implemented data processing pipelines and visualizations using Python and ArcMap.
\end{itemize}

\textbf{Achievements}:
\begin{itemize}[leftmargin=1.5em]
    \item Delivered actionable insights on regional cybercrime dynamics that informed strategic policy recommendations.
    \item Enhanced the overall data visualization framework, resulting in clearer, more accessible reports for interdisciplinary stakeholders.
\end{itemize}

\section{Skills \& Certifications}
\begin{itemize}[leftmargin=1.5em]
    \item Programming \& Data Analysis: Python, C++, ArcMap, statistical modeling, data visualization
    \item Languages: Proficient in English (CET-6: 553)
    \item Technical Tools: Experience with deep learning frameworks (PyTorch) and probabilistic generative models
    \item Analytical Thinking: Possess a solid foundation in mathematics and programming.
\end{itemize}

\end{document}